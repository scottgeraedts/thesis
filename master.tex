\documentclass{cit_thesis}
\usepackage{amsmath, amssymb, braket, units}

\usepackage[mathscr]{euscript}
\usepackage{graphicx}
\usepackage{subfigure}		% This is supposed to be obsolete, superseded by subfig
\usepackage{rotating, multirow, mathtools}
\usepackage{epstopdf, times}

\usepackage[colorlinks=true,linkcolor=blue,citecolor=blue,breaklinks]{hyperref}
\usepackage{breakurl}
\usepackage{color}

\begin{document}
\title{Competing Abelian and non-Abelian topological orders in \texorpdfstring{$\nu = 1/3+1/3$}{filling 1/3+1/3} quantum Hall bilayers}
\author{Scott Geraedts}


\begin{abstract}
Bilayer quantum Hall systems, realized either in two separated wells or in the lowest two sub-bands of a wide quantum well, provide an experimentally realizable way to tune between competing quantum orders at the same filling fraction.
Using newly developed density matrix renormalization group techniques combined with exact diagonalization, we return to the problem of quantum Hall bilayers at filling $\nu = 1/3 + 1/3$.
We first consider the Coulomb interaction at bilayer separation $d$,  bilayer tunneling energy $\Delta_\textrm{SAS}$, and individual layer width $w$,  where we find a phase diagram which includes three competing Abelian phases:  a bilayer-Laughlin phase (two nearly decoupled $\nu = 1/3$ layers);  a bilayer-spin singlet phase;  and a bilayer-symmetric phase. 
We also study the order of the transitions between these phases.
A variety of non-Abelian phases have also been proposed for these systems.
While absent in the simplest phase diagram, by slightly modifying the interlayer repulsion we find a robust non-Abelian phase which we identify as the ``interlayer-Pfaffian" phase.
In addition to non-Abelian statistics similar to the Moore-Read state, it exhibits a novel form of bilayer-spin charge separation.
Our results suggest that $\nu = 1/3 + 1/3$ systems merit further experimental study.
\end{abstract}

\maketitle
\tableofcontents

\end{document}
