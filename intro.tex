\chapter{Introduction}

During the course of my graduate research, I have worked on two sets of projects, which are not closely related to one another. In the first set of projects, I developed a set of analytical and numerical tools for studying a certain class of statistical mechanics models. These models potentially have a number of applications, but the application I focused on was topological phases of bosons. In the end I was able to produce numerically tractable models of topological phases in one, two and three dimensions. This series of projects occupied the majority of my time during my degree. 
More recently, I have been working on projects related to the numerical study of the fractional quantum Hall effect. What these two sets of projects have in common is that they are both studies of topological phases of matter. Therefore this thesis begins with a discussion and motivation for the study of topological phases.

\section{Introduction to Topological Phases}

When discussing topology in condensed matter systems, it is helpful to contrast with the situation before the development of topological concepts. In that time, it was thought that all phases of matter could be uniquely described by their symmetries. For example, the paramagnet-ferromagnet transition could be thought of as a breaking of time-reversal symmetry, while the liquid-solid transition was a breaking of translational symmetry. Given a material known to undergo a phase transition, the task of the condensed matter physicist was to determine the symmetry which was breaking across the transition.

This understanding was overturned by the discovery of the quantum Hall effect in 1980.\cite{vonKlitzing} In the quantum Hall effect, different Hall plateaus are different phases of matter, but they all have the same symmetry. What is different between the different phases turns out to be a topological invariant known as the Chern number. 
%explanation as to why the chern number is topological (from bernevig)

There are many general statements which can be made about symmetry-breaking transitions in condensed matter physics. We now such transitions are described by order parameters, and that they possess critical exponents independent of the microscopic details of the system. We know about the point groups, which tell us all the different ways to break spatial symmetries. We know that breaking continuous symmetries leads to Goldstone modes. These general facts, and others like them, are extremely useful studying condensed matter systems. 

One important direction in the study of topological phases is to establish similar general principles, and great progress has been made in this area over the past several years. In the remainder of this section, I will describe a number of these properties which will be used throughout this work.

The concept of entanglement is crucial to the study of topological phases. Perhaps the simplest system which exhibits entanglement is a pair of spin-1/2 particles which are form a spin singlet. Even if the two particles are well separated in space, it is still possible to the state of one to affect the state of the other. A condensed matter system which contained many spin singlets macroscopically far apart would have a lot of entanglement, whereas a system with few, or very close together, singlets would have little entanglement. As early as 1935 it was realized by Einstein\cite{EPR} that this entanglement leads to highly counterintuitive results, and so it should be no surprise that the matter around us has very little entanglement.

With this in mind, we can divide all states of matter into two classes: short-ranged entangled (SRE) and long-ranged entangled(SRE). If have a state of matter and define two subsystems which are separated by a distance $r$, in an SRE phase the entanglement between the subsystems will decay as $e^{-r}$, while for an LRE system it is either constant or decays as $1/r^{\alpha}$ for some exponent $\alpha$. We will see that long-ranged entangled matter has a variety of very interesting (and perhaps even useful) properties. Topological phases can have either short- or long-ranged entanglement, and these two cases will be discussed separately.

\subsection{Short-ranged entanglement}

Even in the absence of long-ranged entanglement, topological phases of matter can still have exotic properties. In addition, short-ranged entangled phases are arguably simpler to understand than their long-ranged cousins, and so they are a good place to begin any study of topological phase.

Perhaps the most important event in the study of topological SRE phases was the discovery of the quantum spin Hall effect\cite{spinHallReview} and the topological insulator(TI).\cite{KaneHasanRMP,QiZhangRMP}\footnote{The term `topological insulator' is used in the literature to describe a number of different phases, in this work I will use it exclusively to describe the 3D `strong' TI}
Like the integer quantum Hall effect, these phases have topological invariant, which can be computed by integrating the Berry curvature over the Brillioun zone. Unlike the IQHE, this topological invariant is \emph{only quantized in the presence of time-reversal symmetry}. This means that these topological phases only exist when this symmetry is present. This turns out to be a general property of SRE topological phases: they become topologically trivial if certain symmetries are broken. These phases are often called `symmetry protected topological phases' (SPTs).

The topological invariant in both the spin Hall effect and the topological insulator is a total derivative. The upshot of this is that there is no bulk measurement which can distinguish these phases from topologically trivial phases with the same symmetry. There is however exotic edge physics which can distinguish between the two cases. The quantum spin Hall effect has two counterpropagating edge modes (each carrying opposite spin). This cannot happen in a purely one-dimensional system with time-reversal symmetry. The topological insulator has a single Dirac cone, which cannot happen in a two-dimensional time-reversal-invariant system. This is another general property of time-reversal invariant systems: they can host edge physics which cannot exist in a system of one lower dimension.

One may also ask what other SRE topological phases are possible, besides the ones given above. Answers to this question take the form of `classification tables', which, given a symmetry and number of dimensions, tell us how many topological phases there are. Classification tables for free fermion systems\cite{KitaevTable,Ludwig} as well as bosons\cite{WenScience,*WenPRB} have been produced, while the problem of classifying interacting fermions remains an area of active research. Such classification tables provide a roadmap in the search for more examples of SRE topological phases.

\subsection{Long-ranged entanglement}

There are two situations in which matter can have long-ranged entanglement. One is at a critical point, where the entanglement decays algebraically in space. If we are interested in accessing long-ranged entanglement experimentally this is not very useful, as critical points require a lot of fine-tuning and are by definition unstable. The only known way to get a stable phase with long-ranged entanglement is if the phase is topological. 

Like the SRE topological phases discussed above, LRE topological phases can have exotic physics. Unlike SRE phases, LRE phases to not require symmetry to exist. They also have other exotic properties, in particular \emph{fractionalization}. This means that the gapped quasiparticles in a LRE phase can carry fractions of the quantum numbers (such as charge). Much of the physics of LRE phases was first developed during the study of the fractional quantum Hall effect (FQHE). For example, a filling $\nu=1/3$ FQHE system has quasiparticles carrying $1/3$ charge. The spin of the quasiparticles can also become fractionalized. In an SRE system all quasiparticles must be either bosons or fermions, which means that upon interchanging two quasiparticles the phase of the wave function must change by $\pm 1$. In an LRE system quasiparticles (in two dimensions) can be `anyons', and interchanging them can either multiply the wave function by any complex phase (if the anyons are abelian), or put the system into a different degenerate ground state (for non-abelian anyons, which will be explained in more detail below). 
From the anyonic statistics of the quasiparticles we can also show that LRE topological phases have a ground state degeneracy when put on a topologically non-trivial surface(e.g.~a torus). This is because the operations of taking a quasiparticle around the different cycles of the torus do not commute.

\section{Tractable models of topological phases}

	We can see that many general properties of topological phases have been determined. Note that though those properties hold in general, the concepts were first established for specific examples of topological phases, such as the quantum spin Hall effect, topological insulator and fractional quantum Hall effect. To learn about other general properties of topological phases, more specific examples may be needed. Unfortunately additional experimentally realizable topological phases are in short supply. This does not mean that we are out of luck, because much can be learned from simple models which realize topological phases. It is not hard to find examples of simple models which have led to great progress. The group cohomology classification used by Wen to produce classification tables for bosonic systems\cite{WenScience,*WenPRB} can be understood by studying the Haldane chain, which is a simple model of spin-1 particles with Heisenberg interactions. Issues of ground state degeneracy and the relationship with anyonic quasiparticles can be seen in Kitaev's toric code\cite{KitaevHouches}, while the concept of non-abelian anyons can be most easily understood by studying Kitaev's model of a p+ip superconducting wire. The models of Levin and Gu\cite{LevinGu} and Walker and Wang\cite{WalkerWang,BurnellSimon} have also provided many insights.

The models discussed above can all be solved exactly. This has allowed much progress to be made in the study of these models, but there are also reasons to go beyond exactly-solvable cases. For example, one may want to study the effects of interactions which would ruin the solvability of the model. This may done for purely theoretical reasons, or as part of an effort to construct more realistic Hamiltonians for topological phases. One may also be interested in a class of topological phases for which no exactly solvable model can be found. For these reasons, over the course of my graduate research I have spent a lot of time developing an alternative to these exactly solvable models. Instead of constructing models which can be studied exactly, I have constructed models which can be studied in Monte Carlo simulations. 

The models I have studied began as interesting statistical mechanics models consisting of two species of bosons. When bosons of opposite species are exchanged, the wave function gets an overall phase. This means that the bosons are `mutual anyons'. One would expect that models with non-trivial statistics would contain complex Berry phases in their path integral, which would make Monte Carlo simulations impossible. It turns out that it is possible to eliminate this sign problem through formal manipulations to get a model which can be simulated. We also developed sophisticated analytical tools which can be used to study these systems.

Such models can have many applications, for example they can be used to study certain kinds of exotic critical points. The application most relevant to this work is that these models can be used to realize a bosonic version of the quantum Hall effect. Effective field theories\cite{SenthilLevin} and model wave functions\cite{LuVishwanath} for the boson QHE had already been discovered, but our work was the first microscopic model which could realize this phase. In addition, prior work had only discussed the bosonic integer quantum Hall effect, whereas our methods allowed for the study of both integer and fractional boson quantum Hall effects.

I also have constructed a similar model of the bosonic topological insulator. The effective field for this phase is also known\cite{VishwanathSenthil}, but prior to our work a microscopic model had not been found. In addition, our models allow the study of a fractional topological insulator of bosons. Note that the boson IQHE and boson TI are short-ranged entangled topological phases, whereas the fractional versions which we also found are long-ranged entangled.

These models provide a new way to learn about these systems. They can potentially answer questions which other methods cannot answer. For example, in the bosonic quantum Hall effect case we have determined the critical exponents for transitions between different quantum Hall plateaus. 

\section{Topological quantum computing and numerical studies of the quantum Hall effect}

The exotic physics possible in topological phases could potentially lead to a wide variety of applications. One of the most exciting applications is topological quantum computing. To understand what this is, consider a `non-topological' quantum computer, for example one whose qubits consist of cold atoms, each of which can be in two different states. These qubits can be 'decohered' (have their state changed randomly) by local perturbations from the environment. This means that any data stored in these qubits will be lost. To get around this problem, a topological quantum computer has as its qubits a long-ranged entangled topological phase, each of which has multiple degenerate ground states. If the gapped quasiparticles of the system are exotic particles called non-abelian anyons, then exchanging two particles can switch the system between ground states. The key is that the exchange of non-abelian anyons is a non-local process, and so cannot be easily decohered by the environment. 

The search for a condensed matter system which can host non-abelian anyons and therefore act as a topological quantum computer has been a vigorous one in recent years. Various systems, such as possible $p+ip$ superconductors, and semiconductor heterostructures have been proposed. One particularly appealing proposal is to find non-abelian anyons in a quantum Hall system. This began with the work of Moore and Read, who proposed that the observed plateau at $\nu=5/2$ can host non-abelian anyons called Majorana fermions. Since that time, non-abelian phases which can exist at many other filling fractions have been proposed\cite{ReadRezayi,BondersonSlingerland}. Unfortunately, even if a given non-abelian phase is consistent with a given filling fraction, this does not mean that that non-abelian phase is the ground state for the systems prepared at that filling by experimentalists. 

It is therefore an important problem to determine whether the systems realized experimentally are indeed the sought after non-abelian phases. This may be possible experimentally, but as the experiments are likely very difficult it is up to theorists to provide guidance as to which filling fractions are most likely to host non-abelian phases, and under what conditions. Unfortunately, performing this task requires quantitatively estimating the energy of the possible ground states in the strongly interacting quantum Hall problem, and this is very difficult analytically.

In order to determine whether the non-abelian phase is indeed the ground state of a realistic system we therefore turn to numerics. Particularly, these systems can be studied using density matrix renormalization group (DMRG) algorithms\cite{Zaletel}. DMRG is a variational method which works well for one-dimensional, gapped systems. To study a quantum Hall system, we put the system on a cylinder. The computing resources required to study the system scale as $e^L$, with $L$ the circumference of the cylinder. Though this exponential growth is unfortunate, it is in only one of the two dimensions of the system, and this compares well with the rival method, exact diagonalization, which scales as $e^{L^2}$. 

DMRG methods have already been applied to look for non-abelian phases in two possible places, $\nu=5/2$ and $\nu=12/5$. In this work I will discuss my attempt to repeat the same procedure for a bilayer quantum Hall system, where each layer has $\nu=1/3$. One advantage of studying bilayer systems is that the experimentally there are more ways to tune the interactions in the system. By using DMRG on a realistic Hamiltonian I was able to determine that the most straightforward experiment will not see any non-abelian phases. I was however able to address other experimentally relevant questions such as the nature of the abelian topological phases expected, the order of the transitions between them, the effects of having quantum wells of finite width, and the magnetic field needed to spin polarize the system.

In addition, I was able to determine what changes to the Coulomb interaction between electrons would lead to a non-abelian phase, and identify the resulting non-abelian phase as the `interlayer Pfaffian'.

\section{Overview of contributions}

Figure \ref{papers} shows a representation of all of the publications produced during my graduate research. Not all of the publications in this figure will be adequately covered in this report, this figure has been included to show how the projects that I will discuss fit into the other research that I have done. 

The papers are divided into three categories: orange, blue and green. The orange category contains work done to develop the statistical mechanics models of bosons with mutual statistics. The paper marked A in the figure was the first to study such models in sign-free Monte Carlo simulations. It was specialized to the case where the bosons were mutual fermions, and interacted with each other on-site interactions. 
The paper marked B was a related study of a system with only a single species of bosons, but these bosons had `marginally long-ranged' ($1/r^2$) interactions. The order of the Mott insulator-superfluid transition in this system was the subject of some controversy, and we definitively showed that the transition was second order while also developing the numerical techniques needed to handle bosons with arbitrary interactions. 
Paper C also studied bosons with delta function interactions but generalized the mutual statistics of the two species to be any number. (The bosons were `mutual anyons', instead of `mutual fermions') 
Finally paper D generalized our methods to any interaction and any mutual statistics. In particular we focussed on the marginally-long-ranged case, where our analytical tools allowed for the complete establishment of the phase diagram. The results found for the various models studied in these papers will not be discussed in this report. However, the methods used to study these systems proved extremely useful throughout my graduate research, and so these methods are outlined in Chapter \ref{chapter::methods}.

The blue category takes the methods developed in the orange category and applies them to the problem of studying topological phases of bosons. In Paper E, which is discussed in Chapter \ref{chapter::FQHE}, the method are applied to the bosonic topological insulator. We next wanted to apply the same methods to three-dimensional topological phases. To gain some intuition for this we first went backward and developed an exactly solvable model for a class of one-dimensional topological phases, which is described in paper F. In Paper G, which is discussed in Chapter \ref{chapter::SO34D}, we successfully construct a model for the three-dimensional bosonic topological insulator. In paper H we consider a variant of our model for the boson fractional topological insulator, which instead realizes a novel long-range entangled phase of a lattice gauge theory. Finally in paper J we use the model of the boson QHE to study the transitions between different quantum Hall phases.

The green category covers the work on the (fermionic) fractional quantum Hall effect accomplished using DMRG. This category contains only paper I, and is discussed in Chapter \ref{chapter::onethird}
