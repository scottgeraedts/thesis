\chapter{Introduction}

During the course of my graduate research, I have worked on two separate projects, which are not closely related to one another. In the first project, I developed a set of analytical and numerical tools for studying a certain class of statistical mechanics models. These models potentially have a number of applications, but the application I focused on was topological phases of bosons. In the end I was able to produce numerically tractable models of topological phases in one, two and three dimensions. This series of projects occupied the majority of my time during my degree. 
More recently, I have been working on projects related to the numerical study of the fractional quantum Hall effect. What these two sets of projects have in common is that they are both studies of topological phases of matter. Therefore this thesis begins with a discussion and motivation for the study of topological phases.

\section{Introduction to Topological Phases}

When discussing topology in condensed matter systems, it is helpful to contrast with the situation before the development of topological concepts. In that time, it was thought that all phases of matter could be uniquely described by their symmetries. For example, the paramagnet-ferromagnet transition could be thought of as a breaking of time-reversal symmetry, while the liquid-solid transition was a breaking of translational symmetry. Given a material known to undergo a phase transition, the task of the condensed matter physicist was to determine the symmetry which was breaking across the transition.

This understanding was overturned by the discovery of the quantum Hall effect in 1980. In the quantum Hall effect, different Hall plateaus are different phases of matter, but they all have the same symmetry. What is different between the different phases turns out to be a topological invariant known as the Chern number. 
%explanation as to why the chern number is topological (from bernevig)

There are many general statements which can be made about symmetry-breaking transitions in condensed matter physics. We now such transitions are described by order parameters, and that they possess critical exponents independent of the microscopic details of the system. We know about the point groups, which tell us all the different ways to break spatial symmetries. We know that breaking continuous symmetries leads to Goldstone modes. These general facts, and others like them, are extremely useful studying condensed matter systems. 

One important direction in the study of topological phases is to establish similar general principles, and great progress has been made in this area over the past several years. In the remainder of this section, I will describe a number of these properties which will be used throughout this work.

The concept of entanglement is crucial to the study of topological phases. Perhaps the simplest system which exhibits entanglement is a pair of spin-1/2 particles which are form a spin singlet. Even if the two particles are well separated in space, it is still possible to the state of one to affect the state of the other. A condensed matter system which contained many spin singlets macroscopically far apart would have a lot of entanglement, whereas a system with few, or very close together, singlets would have little entanglement. As early as 1935 it was realized by Einstein that this entanglement leads to highly counterintuitive results, and so it should be no surprise that the matter around us has very little entanglement.

With this in mind, we can divide all states of matter into two classes: short-ranged entangled and long-ranged entangled. If have a state of matter and define two subsystems which are separated by a distance $r$, in a short-ranged entangled phase the entanglement between the subsystems will decay as $e^{-r}$, while for a long-ranged entangled system it is either constant or decays as $1/r^{\alpha}$ for some exponent $\alpha$. We will see that long-ranged entangled matter has a variety of very interesting (and perhaps even useful) properties. Topological phases can have either short- or long-ranged entanglement, and these two cases will be discussed separately.

\subsection{Short-ranged entanglement}

Even in the absence of long-ranged entanglement, topological phases of matter can still have exotic properties. In addition, short-ranged entangled phases are arguably simpler to understand than their long-ranged cousins, and so they are a good place to begin any study of topological phase.

Perhaps the most important event in the study of short-ranged entangled phases was the discovery of the quantum spin Hall effect and the topological insulator.\footnote{The term `topological insulator' is used in the literature to describe a number of different phases, in this work I will use it exclusively to describe the 3D `strong' TI}
Like the integer quantum Hall effect, these phases have topological invariant, which can be computed by integrating the Berry curvature over the Brillioun zone. Unlike the IQHE, this topological invariant is \emph{only quantized in the presence of time-reversal symmetry}. This means that these topological phases only exist when this symmetry is present. This turns out to be a general property of short-ranged entangled topological phases: they become topologically trivial if certain symmetries are broken. These phases are often called `symmetry protected topological phases' (SPTs). 

Another 

\subsection{Long-ranged entanglement}


There are two situations in which matter can have long-ranged entanglement. One is at a critical point, where the entanglement decays algebraically in space. If we are interested in accessing long-ranged entanglement experimentally this is not very useful, as critical points require a lot of fine-tuning and are by definition unstable. The only known way to get a stable phase with long-ranged entanglement is if the phase is topolgical. These long-ranged entangled topological phases have a number of exotic propertiies.

\section{Motivation Part 1: The utility of tractable models of topological phases}

\section{Motivation Part 2: Connections to experiment and topological quantum computing}
 

\section{Overview of contributions}

Figure \ref{papers} shows a representation of all of the publications produced during my graduate research.  
